\documentclass[a4paper,12pt]{article}
\usepackage{graphicx}
\usepackage[top=2in,bottom=1in,left=1.25in,right=1.25in]{geometry}

\begin{document}

\large

\title{Apvlv - Alf's PDF Viewer Like Vim}
\author{version 0.0.6}
\date{by Alf $<$naihe2010@gmail.com$>$ \\Apvlv is open source and freely distributable}
\maketitle{}

\begin{center}
\begin{tabular}{lll}
type & :help info$<$Enter$>$      & for infomation \\
type & :q                         & for exit \\
type & :help command$<$Enter$>$   & for command \\
type & :help setting$<$Enter$>$   & for setting \\
\end{tabular}
\end{center}

\newpage

\section{Infomation}

Apvlv is a open source software, and was created by Alf. Now, it is still growing.

Like the name, Apvlv makes people reading their PDF files just like using Vim.

So, Apvlv bindings lots of Vim command and its behaviour is like Vim. For example, $<$Ctrl-f$>$ to forward page, $<$Ctrl-b$>$ to previous page, 'k','j','h','l' to scrolling a page up, down, left or right, and so on. 

And, Apvlv can understand that how many times you want to run the command.

The only thing you need to do is typing the number before the command. For example, typing '50' and $<$Ctrl-f$>$ will go forward 50 pages, typing '30' and $<$Ctrl-b$>$ will go previous 30 pages.

What's more important, Apvlv can understand negative numbers too. If it got '-50' and 'Ctrl-B', PDF document will go FORWARD (preward -50) 50 pages. For example, typing '50G' will goto the 50th page, and typing '-50G' will go to the 50th page from the last page to count.

Do you like it? If yes, continue read the help document.

\newpage

\section{How to start}

\subsection{How to install}

Apvlv is a standard GNU software, so you can install it as normal.

Just need to
\begin{verbatim}
# ./configure
# make
# make install
\end{verbatim}
will install it to /usr/local.

If you got error message like 'No package 'poppler-glib' found', mya be there is not poppler in your OS. You need to install it by yourself.

If you want to install apvlv in a custom path, use './configure --prefix=YOUR CUSTOM PATH' to replace './confiugre' will OK.

\subsection{How to start the binary program}

You can start apvlv by typing 'apvlv' in any terminal which is running under X.

If you got message like 'command not found', make sure the path which apvlv's binary in is in your "PATH" evorment. 

If not, run 'export PATH="/usr/local/bin:\$PATH"' to append the path.

\newpage

\section{Command}

\begin{description}

\item o

display a file chooser dialog to select a PDF file to open.

\item O

select a directory to display

\item R

reload the file which was be read.

\item r

[count] rotate the document page view

\item G

[count] page show

\item gt

next tab show

\item gT

previous tab show

\item $<$PageDown$>$

\item $<$C-f$>$

[count] forward page 

\item $<$PageUp$>$

\item $<$C-b$>$

[count] preward page 

\item $<$C-d$>$

[count] half forward page 

\item $<$C-u$>$

[count] half previous page 

\item H

scroll to head

\item M

scroll to middle

\item L

scroll to last

\item $<$C-p$>$

\item $<$Up$>$

\item k

[count] page scroll up

\item $<$C-n$>$

\item $<$Down$>$

\item $<$C-j$>$

\item j

[count] page scroll down

\item $<$BackSpace$>$

\item $<$Left$>$

\item h

[count] page scroll left

\item $<$Space$>$

\item $<$C-l$>$

\item l

[count] page scroll right

\item f

If Apvlv is in full screen mode, switch to normal mode. Other wise, switch to full screen.

\item zi

zoom in the page

\item zo

zoom out the page

\item m

Mark the read position to the followed char. \\
That is, press 'ma' will mark the current position to a, and you can return here by press 'a.

\item '

Goto the position which point by the followed char. \\
The next key be pressed must had been marked before. 

\item ''

Return to the presious position

\item $<$C-w$>$ q
\item $<$C-w$>$ $<$C-Q$>$

Close the current window.

\item $<$C-w$>$ $<$C-w$>$

move to the next window.

\item $<$C-w$>$ k

move to the up window.

\item $<$C-w$>$ j

move to the down window.

\item $<$C-w$>$ h

move to the left window.

\item $<$C-w$>$ l

move to the right window.

\item $<$C-w$>$ -

make the current window smaller.

\item $<$C-w$>$ +

make the current window bigger.

\end{description}

And, all of the commands can be map to other keys.

For examble, if you put 
'map $<$C-n$>$ o'
in your .apvlvrc.

Then, you type $<$Ctrl-n$>$ will run the 'o' command, that is, show a file chooser dialog to select a PDF file to open.

Becareful, You Should not map a command to itself.

So, 'map god god' will make the application to loop test the command 'god' for ever.

\newpage

\section{Setting}

Apvlv support these settings. You can set any of them in .apvlvrc.

\begin{description}

\item fullscreen=yes/no

If set yes, apvlv will in full screen mode after starup.

\item width=$<$int$>$

When fullscreen is not set to yes, this is the window width.

\item height=$<$int$>$

When fullscreen is not set to yes, this is the window height.

\item defaultdir=$<$path$>$

When display the open dialog, this is the default directory.

\item zoom

This argument has 4 types.

\begin{enumerate}

\item normal

the default zoom value will be set by the application itself.

\item fitwidth

the default zoom value will let the page width eque to the window width

\item fitheight

the default zoom value will let the page height eque to the window height

\item custom

set a custom value. like zoom=1.0, zoom=1.2, zoom=0.8, ...

\item continuous=yes/no

set a pdf page continuous or not

\end{enumerate}

\end{description}

\newpage

\section{Prompt Command}

\begin{description}

\item :h[elp]

display the help pdf document.

\item :h[elp] info

display the help pdf document about the introduction.

\item :h[elp] command

display the help pdf document about the command.

\item :h[help] setting

display the help pdf document about the setting in the .apvlvrc

\item :h[elp] prompt

display the help pdf document about the prompt command.

\item :q[uit]

close the current window.\\
If the window is the top level window, quit the program.

\item o[pen] 'filename'

open a file which named 'filename'.

\item doc 'filename'

load the 'filename' to current window.

\item :TOtext

translate the current page to a text file.

\item :pr[int]

print the current document.

\item :TOtext 'filename'

translate page to a text file named 'filename'.

\item :tabnew

create a new tab

\item :sp

split the current window into two windows.

\item :vsp

split the current window into two horizon windows.

\item :fp

\item :forwardpage 

go forward some pages.

\item :bp

\item :prewardpage

go preward some page.

\item :g

\item :goto

go to a page.

\item :set [no]cache

set if use cache module. If you don't feel scroll slowly, you don't need set this to on.

\item :z[oom]

zoom to a value

\end{description}

\newpage

\section{About}

\large
\begin{tabular}{l}
Website: http://apvlv.googlecode.com \\
Author: Alf $<$naihe2010@gmail.com$>$ \\
Blog: http://naihe2010.cublog.cn
\end{tabular}

\end{document}
